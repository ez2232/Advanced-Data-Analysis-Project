\documentclass[11pt]{article}

\usepackage{color}
\usepackage{caption}
\usepackage{amsmath}
\usepackage{amsthm}
\usepackage{amsfonts} 
\usepackage{amssymb}
\usepackage{multirow}
\usepackage[pdftex]{graphicx}
\usepackage{epsfig}
\usepackage{latexsym}
\usepackage{enumerate}
\usepackage{tabularx}
\usepackage{booktabs}
\usepackage{fullpage}
\usepackage{algorithm}
\usepackage{centernot}
\usepackage[noend]{algpseudocode}

\newtheorem{thm}{Theorem}
\newtheorem{problem}[thm]{Problem}
\newenvironment{claim}[1]{\par\noindent\textit{Claim:}\space#1}{}

\newcommand{\R}{\mathbb{R} }
\newcommand{\Q}{\mathbb{Q} }
\newcommand{\Z}{\mathbb{Z} }
\newcommand{\N}{\mathbb{N} }
\newcommand{\C}{\mathbb{C} }
\newcommand{\Prob}[1]{ \mathbb{P} \left[ #1 \right] }
\newcommand{\Given}{\middle|}
\newcommand{\overbar}[1]{\mkern 1.5mu\overline{\mkern-1.5mu#1\mkern-1.5mu}\mkern 1.5mu}

\begin{document}

\title{Visualizing Hillary Clinton's Emails}

\author{
  Yihe Chen \\
  \texttt{yc3076}
  \and 
  Palmer Lao \\
  \texttt{pol2105}
  \and
  Daitong Li \\
  \texttt{dl2991}
  \and
  Ziyue Shuai \\
  \texttt{zs2285}
  \and
  Eric Zhang \\ 
  \texttt{ez2232}
}

\date{\today}
\maketitle


The social network and summarized text data is combined into a visualization that can be displayed in RShiny. The RShiny dashboard uses the packages visNetwork and igraph. visNetwork is an R package for visualizing networks that allows users to create dynamic and interactive visualizations. Users can physically manipulate edges and nodes in a network graph, select and view nodes by group, and customize the appearance of the network with colors and images. igraph is a set of fast and easy to use network analysis tools in R that allows the user to quickly set up and define network graph relationships that can then be ported over to visNetwork for visualization.

\begin{figure}
	\includegraphics[width=\linewidth]{imagefile}
	\caption{Insert caption}
	\label{Insert label}
\end{figure}

\end{document}