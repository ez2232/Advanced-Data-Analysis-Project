Many similarity functions \cite{textrank-sim-var} as well as the one we implemented rely on metrics defined on bag-of-words representations of text.
While the bag-of-words representation is a popular one, it's well known to have many deficiencies that are exacerbated by carelessly processed text.

One consequence of using a bag-of-words is that words that are not a character-for-character match will not be counted as the same.
Much of our text cleaning effort is dedicated to ensuring that words that are indicative of sentence similarity are mapped to the same word, and that words that are not indicative of sentence similarity are removed.

The very first part of cleaning a sentence is simple: we set all of the characters to lowercase, as the capitalization of a word should, in most cases, not change its meaning.

Similarly, we remove any nonalphanumeric symbols from our text, as it seems that the method used to extract the text from the e-mails into a database left many wayward symbols (e.g.\verb!\n!, the newline symbol).
Any nonalphanumeric symbols were turned into spaces, and any extraneous whitespace was subsequently deleted.

The next transformation we apply is removing stop words.
Stop words are words that almost every valid sentence in the English language contains, such as ``the'', ``a'', ``as'', or ``for''.
Because stop words are so common, leaving them in the sentences would likely inflate the similarity between many bags-of-words.
Leaving stop words in our text would effectively dilute our measure of how important a sentence is.

Our last problem is that different conjugations of words such as ``run'' and ``running'' will be counted as different although they ostensibly are referring to the same activity.
We can attempt to mitigate this problem by applying a popular text cleaning procedure called stemming \cite{willett2006porter}, which attempts to reduce all words to their roots.
For example, ``running'' would be reduced to ``run'', and ``run'' would stay the same.
This allows us to measure text like ``I am running for president'' and ``a run for office'' as similar despite the fact that (after removing stopwords) none of the words are a character-for-character match.
The \texttt{R} implementation we used is from the \texttt{tm} package, which implements the standard Porter stemming algorithm.
