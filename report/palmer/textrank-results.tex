TextRank proves to be quite useful when summarizing long e-mails.
When prompted for the top 3 sentences of the following text, TextRank picks out the bolded sentences:

\begin{quotation}
`Assume the following two pieces from Dawn in Karachi reached you through other channels, but just to be sure.
The first is exactly the bad story I was worried about.
The second is the Administration's attempt to shoot it down.
\textbf{Bottom line (another four-letter word): ``Whew!''
FIRST STORY
Talks under way for N-deal with US: Haqqani
By Zulciernain Tahir
Monday, 15 Feb, 2010 I 06:02 AM PST I
LAHORE: Pakistan's Ambassador to US Husain Haqqani has said the government has started negotiating with the United
States for an agreement on nuclear technology.}
``The US is not sceptical about our nuclear programme. \textbf{Talks between Pakistan and the US for cooperation on atomic
programmes are under way and we want the US to have an agreement with us like the one it had with India on civil
nuclear technology,'' Mr Haqqani said at a reception hosted by Punjab Governor Salmaan Taseer on Sunday.}
He said Pakistan would get 16 latest F-16 aircraft in June. He said although the expectations of Pakistan and the US with
each other usually did not fulfil, both were indispensable for each other.
``We have to largely depend upon the US for our defence related matters.
India is our main concern as it is buying weapons worth \$100 billion from five countries, including China, and to balance
it our relations with the US are very significant,'' he said and added that India had 5,500 tanks and there was a question
against whom they would be used. ``We cannot be assured by statements that India will not wage a war against us.''
Giving a reason as to why Pakistan had to look towards the US for enhancing its military capacity and capability, Mr
Haqqani said the European countries did not offer soft terms for buying weapons. ``Ties with the US are important for a
secure, stable and prosperous Pakistan.''
Mr Haqqani said Pakistan had also made it clear on the US that it should ensure a strengthened and Islamabad-friendly
regime in Kabul before leaving.
He said Pakistan had sought drone technology from America. ``On one hand our innocent people are losing their lives
while on the other Taliban leaders like Baitullah Mehsud get killed in such attacks,'' he said.
Mr Haqqani said the US wanted to strengthen democracy in Pakistan and aid under the Kerry-Lugar Bill had started
coming from January.
In reply to a question, he said Pakistan's embassy in the US was working on diplomatic and legal aspects in Dr Aafia
Siddiqui case and was making efforts for securing her release or transfer of her case to a Pakistani court.
\textbf{SECOND STORY
No nuclear deal with Pakistan, says US
By Anwar lqbal
Monday, 01 Mar, 2010 I 07:15 AM PST I
WASHINGTON: The Obama administration has told Pakistan it would not get an atomic power plant or a civilian nuclear
deal from the United States.}
A senior US official, while briefing Indian journalists in Washington, said the United States was working closely with
Pakistan to help meet its growing energy needs.
``But nuclear power is not currently part of our discussions,'' and the United States had conveyed its decision to Pakistan,
the official said.
He said the administration had also told Pakistan that ``there is no way they can get a civilian nuclear deal similar to the
one the Obama administration has signed with India.''
The Indo-US civilian nuclear deal, the official said, was ``specific to India only and there is no thinking going on in the
administration to create a template for it.''
\end{quotation}

In our opinion, the three sentences are fairly representative of the entire text.
The entirety of the e-mail is about two stories, and the summary produced captures both headlines as well as another sentence from one of the articles.